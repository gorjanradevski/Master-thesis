\documentclass[master=mai,masteroption=ecs]{kulemt}
\setup{title={Siamese multi-hop attention for
image-text matching},
  author={Gorjan Radevski},
  promotor={Prof. dr. Marie-Francine Moens},
  assessor={Prof. dr. Tinne Tuytelaars \\
            Guillem Collel Taleda},
  assistant={Guillem Collel Taleda}}
% The following \setup may be removed entirely if no filing card is wanted
\setup{filingcard,
  translatedtitle=Siamese multi-hop aandacht voor beeld-tekst matching,
  udc=621.3,
  shortabstract={In this thesis, the problem of image-text matching is studied. I leverage the fact that usually, multiple entities are present in the modalities, so a model that can focus on different aspects of each modality should overpower simpler models. Moreover, the majority of the approaches that tackle the image-text matching problem are used as a black box, and the correspondence between the matched image and text remains vague. Lastly, the image-text matching datasets are small, and the training data samples are scarce. Because of that, a heavy emphasis is placed on transferring knowledge from models trained on millions of data samples on different tasks. In this thesis, I present \textit{Siamese multi-hop attention}, a deep learning architecture that overcomes the obstacles mentioned above when performing image-text matching. The \textit{Siamese multi-hop attention} model is evaluated on the cross-modal retrieval task, which is the principal method of evaluating image-text matching models. To do so, I perform experiments on a variety of cross-modal retrieval benchmark datasets that include small ones such as \textit{Pascal1k}, \textit{Flickr8k} as well as medium to large ones such as \textit{Flickr30k} and achieve results competitive to the current state-of-the-art. The code developed is made available at: \href{https://github.com/gorjanradevski/SMHA}{https://github.com/gorjanradevski/SMHA}}}
% Uncomment the next line for generating the cover page
%\setup{coverpageonly}
% Uncomment the next \setup to generate only the first pages (e.g., if you
% are a Word user. 
%\setup{frontpagesonly}

% Choose the main text font (e.g., Latin Modern)
\setup{font=lm}

% If you want to include other LaTeX packages, do it here. 
% Float to bind tables to their sections
\usepackage{float}
% Packages to display two images within a figure one next to the other
\usepackage{graphicx}
\usepackage{caption}
\usepackage{subcaption}
\usepackage{natbib}
\bibliographystyle{abbrvnat}
\setcitestyle{numbers,open={[},close={]},comma}
% Finally the hyperref package is used for pdf files.
% This can be commented out for printed versions.
\usepackage[pdfusetitle,colorlinks,plainpages=false]{hyperref}

%%%%%%%
% The lipsum package is used to generate random text.
% You never need this in a real master's thesis text!
\IfFileExists{lipsum.sty}%
 {\usepackage{lipsum}\setlipsumdefault{11-13}}%
 {\newcommand{\lipsum}[1][11-13]{\par And some text: lipsum ##1.\par}}
%%%%%%%

%\includeonly{chap-n}
\begin{document}

\begin{comment}
\begin{preface}
  I would like to thank everybody who kept me busy the last year,
especially my promoter and my assistants. I would also like to thank the
  jury for reading the text. 
\end{preface}
\end{comment}

\tableofcontents*

\begin{abstract}
In this thesis, the problem of image-text matching is studied. I leverage the fact that usually, multiple entities are present in the modalities, so a model that can focus on different aspects of each modality should overpower simpler models. Moreover, the majority of the approaches that tackle the image-text matching problem are used as a black box, and the correspondence between the matched image and text remains vague. Lastly, the image-text matching datasets are small, and the training data samples are scarce. Because of that, a heavy emphasis is placed on transferring knowledge from models trained on millions of data samples on different tasks. In this thesis, I present \textit{Siamese multi-hop attention}, a deep learning architecture that overcomes the obstacles mentioned above when performing image-text matching. The \textit{Siamese multi-hop attention} model is evaluated on the cross-modal retrieval task, which is the principal method of evaluating image-text matching models. To do so, I perform experiments on a variety of cross-modal retrieval benchmark datasets that include small ones such as \textit{Pascal1k}, \textit{Flickr8k} as well as medium to large ones such as \textit{Flickr30k} and achieve results competitive to the current state-of-the-art. The code developed is made available at: \href{https://github.com/gorjanradevski/SMHA}{https://github.com/gorjanradevski/SMHA} 
\end{abstract}

% A list of figures and tables is optional
\listoffigures
\listoftables
% If you only have a few figures and tables you can use the following instead
% \listoffiguresandtables
% The list of symbols is also optional.
% This list must be created manually, e.g., as follows:
\chapter{List of Abbreviations and Symbols}
\section*{Abbreviations}
\begin{flushleft}
  \renewcommand{\arraystretch}{1.2}
  \begin{tabularx}{\textwidth}{@{}p{20mm}X@{}}
    \textit{SMHA}   & Siamese multi-hop attention \\
    \textit{SOTA}   & State of the art \\
    \textit{LSTM}   & Long-short term memory \\
    \textit{GRU}    & Gated recurrent unit \\
    \textit{Word2vec} &  Word to vector \\
    \textit{ReLU} & Rectified-linear unit \\
    \textit{ELMo} & Embeddings from language model \\
    \textit{tanh} & Hyperbolic function \\
    \textit{ResNet} & Residual network \\
    \textit{VGG} &  Visual Geometry Group's neural network \\
    \textit{CNN} &  Convolutional neural network \\
    \textit{R-CNN} &  Region-convolutional neural network \\
    \textit{AlexNet} & Convolutional neural network designed by Alex Krizhevsky \\
    \textit{max(x,y)} & Maximum of \textit{x} and \textit{y}
  \end{tabularx}
\end{flushleft}
\section*{Symbols}
\begin{flushleft}
  \renewcommand{\arraystretch}{1.2}
  \begin{tabularx}{\textwidth}{@{}p{20mm}X@{}}
    \textit{p}    & Probability \\
    $\prod_{i}^{n}$ & Product from \textit{i} to \textit{n} \\
    $\sum_{i}^{n}$ & Sum from \textit{i} to \textit{n} \\
    $\sigma$ & Activation function \\
    $s$ & Scoring function \\
    $\epsilon$ & A small number
  \end{tabularx}
\end{flushleft}

% Now comes the main text
\mainmatter

\chapter{Introduction}
\label{cha:intro}

\section{Image-text matching}
In this thesis, the problem of image-text multimodal matching is studied. Image-text matching plays an essential role for sentence retrieval (retrieving sentences given an image query) and image retrieval (retrieving images given a sentence query). Recently, a variety of methods that attempt to solve the image-text matching problem have been proposed. The methods generally fall into two categories:
\begin{itemize}
    \item One-to-one matching \cite{kiros2014unifying, frome2013devise, ma2015multimodal, mao2014explain, klein2015associating, faghri2017vse++, wang2018learning, socher2014grounded}: Methods where a global representation is extracted from both modalities and matched using a similarity measure.
    \item Many-to-many matching \cite{nam2017dual, lee2018stacked, karpathy2014deep, karpathy2015deep, huang2017instance}: Methods where a sequence of local representations are extracted from both modalities, and the global similarity is computed as an average of the local similarities.
\end{itemize}
In this thesis, I argue that the methodology employed by the one-to-one matching methods is not sophisticated enough to consider all alignments between an image and a sentence. On the other hand, the many-to-many methods take upon an approach which amplifies the complexity of the image-text matching problem.\endgraf
Finally, the \textit{Siamese multi-hop attention} model tries to bridge the gap between the one-to-one and the many-to-many matching methods. The modus operandi of \textit{SMHA} is to leverage the simplicity of the one-to-one matching methods while taking advantage of the comprehensive methodology the many-to-many matching methods employ. To do so, \textit{SMHA} employs a multi-step process in an iterative manner, where a global representation is extracted from both modalities multiple times. Each time the representations are extracted, \textit{SMHA} focuses on different aspects of the modalities, thus searching for all possible correlations that may exist between the image and the text.

\section{Attention mechanism}

\begin{comment}
The essential component of the \textit{Siamese multi-hop attention} is the attention mechanism \cite{bahdanau2014neural, xu2015show}. Earlier variants of the attention mechanism applied in neural machine translation \cite{bahdanau2014neural} and in neural caption generation \cite{xu2015show} limit the number of attention hops to a single one. A later extension is the structured self-attention \cite{lin2017structured}, which is an augmented version of the standard attention mechanism that extends the single-hop attention to a multi-hop one. In this thesis, I start from the premise that multi-hop attention is crucial for successful image-text matching. In regards to that, the contributions of this thesis are the following:
\end{comment}


The essential component of the \textit{Siamese multi-hop attention} is the attention mechanism \cite{bahdanau2014neural, xu2015show}. Earlier variants of the attention mechanism were applied in neural machine translation \cite{bahdanau2014neural} and in neural caption generation \cite{xu2015show}, where attention is applied to associate two sources of information. Later, \citet{cheng2016long, parikh2016decomposable, yang2016hierarchical} implemented self-attention, an attention variant used in conjunction with a recurrent neural network to compute a single linear combination of the hidden states, and thus represent the sequence. Structured self-attention \cite{lin2017structured}, is an augmented version of self-attention which argues that computing a single linear combination is not sufficient, and extends the single-hop attention to a multi-hop one. In this thesis, I start from the premise that multi-hop attention is crucial for successful image-text matching. In regards to that, the contributions of this thesis are the following:
\begin{itemize}
    \item I expand the \textit{structured self-attention} \cite{lin2017structured} for the image-text matching problem. I conduct a series of ablation studies to prove that multi-hop attention is necessary to find all alignments between an image and a sentence.
    \item I empirically prove that using multiple hops of attention on its own is not enough for the model to leverage the increased capacity. As a result, a penalization term \cite{lin2017structured} is implemented to enforce diversity between the attention hops, and push the model towards fully utilizing the multi-hop attention.
\end{itemize}

\section{Siamese networks}
A siamese neural network \cite{bromley1994signature} consists of twin networks that have tied weights. The siamese neural networks were firstly used by \citet{bromley1994signature} to solve the challenge of signature verification, and now are the de facto deep neural network architecture for face verification \cite{taigman2014deepface}. Due to the tied weights, it is guaranteed that two highly similar inputs can not be embedded in different locations in the joint latent space. Moreover, the twin networks compute the same function, so even if the inputs are switched between the twin networks, the output remains the same, thus making the siamese neural network symmetrical. However, the success of the siamese neural networks has been purely related to unimodal matching, and to the best of my knowledge, no study has attempted to leverage a siamese architecture for multimodal matching. On the other hand, the contribution of this thesis is the following:
\begin{itemize}
   \item The two branch \textit{multi-hop attention} is merged in one attention module trained jointly for both modalities. Namely, the \textit{Siamese multi-hop attention} ties the attention weights for both the image and sentence pathway of the model.
   \item  An ablation study is carried out to demonstrate that the less memory demanding \textit{siamese multi-hop attention} performs  on par with the \textit{multi-hop attention}, and learns to pay attention to both modalities at once.
\end{itemize}


\section{Transfer learning}\label{sec: transfer}
Transfer learning is a method where weights from a deep learning model trained on task \textit{A}, involving a feature space $X_A$ and a label space $Y_A$, are transferred to a different deep learning model to perform on task \textit{B} with feature space $X_B$ and label space $Y_B$ where $A \neq B$. By doing so, the receiving model can make use of the knowledge from the model trained on task \textit{A}. Moreover, transfer learning is particularly useful when the domains of task \textit{A} and task \textit{B} are similar, as is the case for the problem being solved in this thesis.\endgraf
Due to the nature and size of the image-text matching datasets, in this thesis, I make heavy use of transfer learning. Compared to other image-text matching approaches where knowledge is transferred to the image encoder, here knowledge is transferred from pre-trained models to both the image and sentence encoder.\endgraf
To summarize, with the addition of transfer learning to the sentence encoder branch, the final contributions of this thesis are:
\begin{itemize}
    \item Compared to other methods for image-text matching, where knowledge from pre-trained models is transferred only to the image encoder branch, \textit{Siamese multi-hop attention} makes use of transfer learning on both the image encoder and the sentence encoder branch.
    \item The \textit{Siamese multi-hop attention} model, with the addition of transferred knowledge to the sentence encoder branch, demonstrates robustness and achieves results competitive to the state-of-the-art on \textit{Pascal1k}, \textit{Flickr8k}, and \textit{Flickr30k} datasets. Furthermore, the \textit{Siamese multi-hop attention} model is interpretable and can visualize the multi-step process it partakes to embed the modalities in a joint latent space.
\end{itemize}

\chapter{Related work}
\label{related_work}
In this section I will do a walk-through the different cross-modal retrieval methods that have performed their experiments on the \textit{Pascal1k}, \textit{Flickr8k} and \textit{Flickr30k} datasets. In general, the methods can be classified into two categories, namely one-to-one matching methods and many-many matching methods. In the former, a global representation is extracted from both modalities and then matched using a similarity measure whereas in the latter, a sequence of local representation is extracted from both modalities and the global similarity between is computed as an average of the local similarities. 
\section{One-to-one matching methods}

\textbf{Semantic dependency tree recurrent neural network} \cite{socher2014grounded} leverages a dependency tree parser in order to extract a hidden representation from the sentence. Moreover, in \textbf{SDT-RNN} the hidden representation from the image is extracted from a convolutionla neural network pretrained on \textit{ImageNet} \cite{krizhevsky2012imagenet}. Lastly, the objective function matches the global objective function from \cite{karpathy2014deep} as well as the mismatching part of the objective function used in this thesis.\endgraf

\textbf{Deep visual semantic embedding} \cite{frome2013devise} to the best of my knowledge, is one of the first methods to leverage transfer learning on both the image encoder and text encoder branches. In \textbf{DeViSE}, each of the words of the sentence are encoded as one-hot vectors which are later embedded using the \textit{Skip-gram} \cite{mikolov2013distributed} model. Then, the sentence representation is obtained by taking the mean vector from all of the word vectors. On the other hand, a hidden representation of the image is obtained using a convolutional neural network pretrained on \textit{ImageNet} \cite{krizhevsky2012imagenet}. Because both of the image embedding and the sentence embedding are in their corresponding latent spaces, each of the modality embeddings are projected into a joint latent space by a one-layer neural network which is followd by the objective function to be minimized. The mismatching part of the objective function used within this thesis matches the one used in \cite{frome2013devise}.\endgraf

\textbf{Associating Neural Word Embeddings with Deep Image Representations
using Fisher Vectors} \cite{klein2015associating} is a method that achieves \textit{SOTA} results on the Pascal1k on most of the metrics. \textbf{FV} is a method that encodes the sentence as a set of word2vec \cite{mikolov2013distributed} vectors. The image embedding on the other hand, is taken from a convolutional neural network pretrained on \textit{ImageNet}. The novelty in \cite{klein2015associating} is that the sets are converted to a Fisher vector using Gaussian Mixture Model, a Laplacian Mixture Model or Hybrid Gaussian-Laplacian Mixture Model. Finally, the scoring function computed using the cannonical correlation analysis.\endgraf

\textbf{Multimodal neural language model} introduced by \cite{kiros2014unifying}, is one of the methods that indicates that a simple model may actually perform really well on the cross-modal retrieval task. The \textbf{MNLM} extracts image embedding from \cite{krizhevsky2012imagenet} together with the sentence embedding which are obtained from the last-hidden state of an \textit{LSTM}\cite{hochreiter1997long}. Once both modalities are encoded by a separate branch, the encodings are projected into a joint latent space. The objective function optimized matches the global alignment objective from \cite{karpathy2014deep, karpathy2015deep} as well as the matching loss of this thesis.\endgraf

\textbf{Multimodal recurrent neural networks}\cite{mao2014explain} is a recurrent neural network based language model that given an image encoding and a sentence, it outputs a probability of the occurrence for each of the words in the sentence. The \textbf{m-RNN} during training is conditioned on the image embedding from a VGG16/19 \cite{simonyan2014very}, and it is trained to minimize the cross-entropy of the next ground truth word in a sequence. During inference, in case the task at hand is sentence retrieval given an image, the \textbf{m-RNN}, will output a probability distribution for each of the sentences in the dataset conditioned on the image embedding. On the other hand, if the task being solved is image retrieval given a sentence, the \textbf{m-RNN} will provide a probability distribution on the query sentence, conditioned on all of the images in the dataset. Therefore, in both cases the output can be seen considered as a ranked list.\endgraf

\textbf{Visual semantic embeddings++} \cite{faghri2017vse++} is an improvement of \cite{kiros2014unifying}. The \textbf{VSE++} takes inspiration from \cite{schroff2015facenet}, and leverages training the model on the hard negatives instead of on the semi-hard and hard negatives. Compared to \cite{kiros2014unifying}, in \textbf{VSE++} they leverage the use of a better image encoder such as \textit{Resnet152} \cite{he2016deep}, instead of \textit{VGG} \cite{simonyan2014very}. The novelty of \cite{faghri2017vse++} is that for each anchor image and a positive sentence in the batch, the hardest negative contrastive sentence selected and vice versa. This is motivated by the fact that the hardest negatives determine the success for the \textit{Recall@1} metric. Consequently, by training only on the hardest negatives, \cite{faghri2017vse++} obtain significant improvement especially on the \textit{Recall@1} metric.\endgraf

\textbf{Multimodal Convolutional Neural Network} \cite{ma2015multimodal} is a multi-modal matching model that leverages convolutional neural networks only for both encoding the modalities and matching them. Namely, a convolutional neural network \cite{simonyan2014very} pretrained on \textit{ImageNet} \cite{deng2009imagenet} is used to encode the image. Then, another \textit{matching} convolutional neural network, takes the word/sentence representation together with the image representation and produces a joint representation. Lastly, a feed-forward neural network is used to translate the joint representation into a matching score between the image and the sentence. The whole architecture is jointly trained by backpropagation \cite{rumelhart1985learning} to optimize the contrastive loss objective function.\endgraf

\textbf{Embedding network} \cite{wang2018learning}, or shortly \textbf{EmbeddingNet} leverages a bidirectional recurrent neural network to encode the sentence into a fixed size vector, and a \textit{VGG} \cite{simonyan2014very}, to obtain a hidden representation of the image. Afterwards, two branches that share the same structure but have separate weights project both of the encoded modalities in a joint hidden space followed by a L2 normalization. Each of the branches has an architecture that consists of a fully connected layer followed by a ReLu \cite{nair2010rectified} non linearity, which is in turn followed by a projection layer in the joint embedding space. The objective function minimized by backpropagation aligns with the matching loss used within the scope of this thesis.\endgraf

\section{Many-to-many matching methods}
\textbf{Deep fragment embeddings} \cite{karpathy2014deep} is a method that makes an explicit assumption that images are complex structures which have multiple entities in them. The sentences however, are soft descriptions about the contents of the image. Therefore, in order to reason about their similarity, \textbf{DFE} \cite{karpathy2014deep} breaks both the image and the sentence into several fragments. Namely, to break the image into image fragments the top 19 detected regions are extracted using an R-CNN \cite{girshick2014rich}. On the other hand, the sentence is broken down into sentence fragments by considering the edges of a dependency tree. Then, every sentence fragment constitutes of a dependency triplet $(R, w_1, w_2)$, where the dependency triplet is mapped into the embedding space by a two-layer neural network. Th3 objective function incorporates both a fragment alignment objective as well as a global objective, where the global objective function matches the mismatching part of the objective function used in this thesis.\endgraf

\textbf{Deep visual-semantic alignments model} \cite{karpathy2015deep}, builds on top of \textbf{DFE} \cite{karpathy2014deep}. It alleviates the bottleneck of extracting the word embeddings using dependency tree parsers, where the context window is fixed, and they use a bidirectional recurrent neural network instead. Moreover, improvements are done on the image embedding branch of the model as well. Namely, in \cite{karpathy2014deep} the top 19 image regions are extracted with an R-CNN \cite{girshick2014rich} which is also the case in \cite{karpathy2015deep}. However, in \cite{karpathy2014deep} for each of these 19 proposed image regions an image embedding is obtained using \cite{krizhevsky2012imagenet} whereas in \textbf{DVSA} \cite{karpathy2015deep} the image region embeddings are obtained using the much deeper VGG network \cite{simonyan2014very}. Afterwards, following \textbf{DFE}, both a fragment and a global alignment are computes as an objective function.\endgraf

\textbf{Selective multimodal LSTM} \cite{huang2017instance} is a model architecture that leverages attention \cite{bahdanau2014neural} in order to find the best alignment between the image embedding and the sentence embedding. \textbf{sm-LSTM}, follows a three step process. In the first step, the instance candidates are obtained using a bidirectional recurrent neural network for the sentence and by evenly dividing the image into regions and passing them through a \textit{CNN}. In the second step, attention is applied on top of the segments from the sentence and the image in order to extract the best possible alignment between them. Once the alignments are found, in the third step the \textbf{sm-LSTM} weights the alignments to describe the image and the sentence and computes the global similarity through a feed-forward neural network. The whole architecture is trained end to end by minimizing the contrastive loss.\endgraf


\textbf{Dual attention networks} \cite{nam2017dual} is a two branch network that leverages attention \cite{bahdanau2014neural} on both the image and the sentence branch. Therefore, the \textbf{DAN} architecture is closely related to the work done within the scope of this thesis. However, the attention mechanism used in \textbf{DAN} and within this thesis are fundamentally different. Namely, in \textbf{DAN}, two vectors of attended values, one for the image and one for the text are updated \textit{K} times. After each update, a dot product similarity is computed between the values. Lastly, \textbf{DAN} sums all the similarities in order to compute the final similarity between the image and the text. The minimized loss function corresponds to \cite{karpathy2014deep, karpathy2015deep, kiros2014unifying} as well as the matching loss used within the scope of this thesis.\endgraf

\textbf{Stacked Cross Attention for Image-Text Matching} \cite{lee2018stacked} is an attention based method for cross-modal retrieval that currently achieves \textit{SOTA} results on the \textit{Flickr30k} dataset. On the image encoder branch, \textit{SCAN} extract the top 19 image regions from a R-CNN. Each of the image regions is then encoded using a convolutional neural network \cite{he2016deep} pretrained on \textit{ImageNet}. The word representations are obtained by taking the average of the forward and backward hidden states of a Bi-RNN which is trained from scratch. After that, attention is applied so that the model can attend on the words of the sentence for each image region or vice versa. Once the image-text or text-image attention is applied, the model uses the cosine similarity objective function combined with the \textit{max-of-hinges} triplet loss \cite{faghri2017vse++}. In particular, in \cite{lee2018stacked} it is reported that an improvement of \textit{48.2\%} on the \textit{Recall@1} is achieved by training only on the hardest negatives within a batch \cite{faghri2017vse++}.
\chapter{The model}
\label{The model}
In this section, I go through the \textit{Siamese multi-hop attention} model in a bottom-up fashion. I start at the lower end of the model, where the raw images and the corresponding sentences enter the model and are met by the data preprocessing and augmentation modules. At the second level of the model architecture are both of the modality encoders, namely the image encoder which is deep convolutional neural network \cite{he2016deep} pretrained on \textit{ImageNet} \cite{deng2009imagenet}, and a sentence encoder which is a language model \cite{peters2018deep} pretrained on the 1 Billion word benchmark dataset \cite{chelba2013one}. Going forward is the attention block where both the outputs from the modality encoders are fed in. The second to last part is the loss computation block where a series of different regularization methods are implemented in addition to the standard matching loss in order to compute the final loss that the model will be trained on. Lastly, the loss is fed into the optimizer block where the gradients are computed and backpropagated through the network \cite{rumelhart1985learning} to update the network weights.

\begin{figure}
  \centering
  \includegraphics[width=70mm]{Images/full_model.png}
  \caption{The full model architecture}
  \label{fig:full_model}
\end{figure}

\section{Data processing}
All of the datasets (with no particular order \cite{rashtchian2010collecting, hodosh2013framing, young2014image}) used to conduct the research in this thesis contain pairs of images and sentences that match one another. Even though the datasets are rather small as per the amount of data that the deep neural nets need to be trained, processing them all at once is not feasible. Because of that, a sequential process has to be implemented where batches of image-sentence pairs are read from the disk, processed, and passed further down the model, so that the training and inference processes can run efficiently. This section in particular sheds light on the processing of the data batches, before they are fed into the model.
\subsection{Image augmentation}
Because the image encoder is \textit{ResNet152} \cite{he2016deep} pretrained on \textit{ImageNet} \cite{deng2009imagenet}, the images are preprocessed in the same way as in \cite{simonyan2014very}. The is due to the fact that the \textit{ResNet152} employs the preprocessing from \textit{VGG16/19}. Because the preprocessing done in \cite{simonyan2014very} comes down to subtracting the \textit{ImageNet} means, the preprocessing here has to be done in the same fashion in order to exclude the chance for any unexpected behaviours. Further more, data augmentation is a series of techniques usually employed when the training data is scarse. This is done in order to include artificial noise in the data and reduce the risk of the model overfitting on the training data. In the standard supervised learning setting $X \rightarrow Y$, the data augmentation techniques are applied on the \textit{X} side, where in each training batch the data is distorted using a series of random transformations. The series of random transformations applied during training the \textit{SMHA}, illustrated on figure \ref{fig:image_augmentation} are:
\begin{itemize}
  \item Random 224x224 centered crop of the image.
  \item Random horizontal flip with 50\% probability.
\end{itemize}
\begin{figure}
  \centering
  \includegraphics[width=100mm]{Images/data_augmentation.png}
  \caption{Image augmentation}
  \label{fig:image_augmentation}
\end{figure}
On the other hand, when performing inference, having noisy data points will be counterproductive. Because of that, in that case, the only transformation done is re-sizing the image to have 224 pixel width and height.
\subsection{Text preprocessing}
In order for the sentence encoder to extract meaningful representations for each of the words in the sentence, the sentences have to cleaned and tokenized. In the cleaning phase, prior to tokenizing the sentence the following transformation are applied to the sentences:
\begin{itemize}
  \item All words in the sentences are lower cased.
  \item All punctuation is removed from the sentences.
\end{itemize}
By doing this, I am ensuring that words which are essentially the same will not be treated as separate words in the vocabulary space. 
\section{Modality encoders}
In this section, I do a walkthrough of both of the modality encoders, namely the image encoder and the sentence encoder.  Since both of the modality encoders heavily rely on transfer learning, in section \ref{sec: transfer} there is a brief introduction of transfer learning which is followed by an explanation of the image encoder in section \ref{sec: image_encoder} and an explanation of the sentence encoder in section \ref{sec: sentence_encoder}.
\subsection{Transfer learning}\label{sec: transfer}
Due to the nature of the cross-modal retrieval datasets, in the scope of this thesis I make heavy use of transfer learning. Compared to other approaches where transfer learning is leveraged on the image encoder branch, here I make use of transfer learning on both the image and text branches to boost the performance further.\endgraf
Transfer learning is the scenario where we are taking a model that has been trained on a task \textit{A}, involving a feature space $X_A$ and a label space $Y_A$, and we are reusing the trained weights from the model to perform on a task \textit{B} with feature space $X_B$ and label space $Y_B$ where $A \neq B$. By doing so, the model that is supposed to perform well on the task \textit{B}, can leverage the knowledge from the model trained on task \textit{A}. Moreover, transfer learning is particularly useful when the domains of the task \textit{A} and task \textit{B} are similar.

\subsection{Image encoder}\label{sec: image_encoder}
The image encoder employed within the model architecture is a deep convolutional neural network \textit{ResNet152}\cite{he2016deep} pretrained on \textit{ImageNet}\cite{deng2009imagenet} and a fully connected layer on top of the \textit{ResNet152} to project the image embedding in the joint latent space. The family of \textit{ResNets} emerged as a way to overcome the degradation problem that occurs when training deeper networks\cite{he2015convolutional}. The degradation problem occurs when, by stacking multiple layers on a neural network, the training accuracy saturates and then decreases rapidly. However, since the accuracy is computed on the training set, this phenomenon cannot be related to overfitting, which is the case where the neural network's performance increases on the training set but decreases on the validation set. The \textit{degradation} phenomenon has been investigated and empirically proven by \cite{he2016deep, he2015convolutional}. Therefore, to overcome the degradation problem, in \cite{he2015convolutional} the residual block is introduced. The main purpose of the residual block is to sidestep learning an explicit mapping between a block of stacked layers, and instead let the stacked layers learn a residual mapping. In order to do so, the original mapping is recast into:

$$F(x) + x$$

by letting the the stacked layers fit the mapping:

$$F(x) = H(x) - x$$

where $H(x)$ is the desired mapping that we originally wanted to learn.
The motivation for doing so is to bypass the \textit{degradation} problem as explained in \cite{he2015convolutional}. Namely, if a neural network with $N$ layers can sufficiently approximate a desired function $F(x)$, then a neural network with $N + M$ layers where the newly added stack of layers are constructed as identity mappings, should approximate the same function. Based on this hypothesis, it follows that the newly added stack of layers has difficulties approximating the identity function. Therefore, if the identity mapping between the layers is desired, the neural network can successfully learn to push its weights towards zero and thus fit the identity mapping.\endgraf
The building block of the \textit{ResNet} family of neural networks is:

$$y = W_2 \sigma(W_1x) + x$$

Where $W_1$ and $W_2$ are the weight matrices to be learned, $x$ are the inputs and $\sigma$ denotes the \textit{ReLU}\cite{nair2010rectified} activation function.\endgraf


During training, the \textit{ResNet152} is kept fixed, while a feature vector from the image is extracted from the last convolutional layer before the logits. That results in a $7\times7\times2048$ dimensional representation of the encoded image. Namely, the activation map of the last convolutional layer of the \textit{ResNet152} results in an output of 2048 filters with a width and height of 7.
The motivation for doing such transfer learning\cite{yosinski2014transferable} is that, it has been empirically proven when a convolutional neural network is trained on a dataset of images, the first layer of the neural network learns features that resembles a Gabor filters or color blobs\cite{yosinski2014transferable} which are easily transferable from one image domain to another. Since the learned features at the first few layers of the neural networks are invariant of the objective function and the dataset, these features are considered as general and can be reused on different tasks. On the other hand, the layers that reside at the end of the neural networks tend to specialize on the training objective. Consequently, the typical transfer learning scenario is to obtain the first few layers of a neural network trained on a bigger dataset, and to append these layers on a new neural network, that will be trained on a specific task. The new, task-specific neural network will include other layers, which will be randomly initialized and trained to make use of the extracted features and map them to its corresponding task. In this particular case, the new neural network is exactly one fully connected layer that projects the \textit{ResNet152} features.\endgraf
Figure \ref{fig:image_encoder} illustrates the image encoder branch of the model. On the lower end, the augmented image is provided as input and the \textit{ResNet152} extracts the image features $(S_1, S_2, S_3,...,S_n)$. Each $S_i$ is of size \textit{2048} and holds a single activation pixel from all of the \textit{2048} filters from the \textit{ResNet152} and \textit{n = 49} which corresponds to the flattened activation map size. After projecting number of filters in the joint space through the fully connected layer, the final output from the image encoder is a list of hidden states $(H_1, H_2, H_3,...,H_n)$ where \textit{n} remains equal to 49 and the size of each $H_i$ corresponds to the size of the joint space.

\begin{figure}
  \centering
  \includegraphics[width=45mm]{Images/image_encoder.png}
  \caption{The image encoder branch}
  \label{fig:image_encoder}
\end{figure}


% SENTENCE ENCODER%


\subsection{Sentence encoder}\label{sec: sentence_encoder}
The input to the sentence encoder is a sequence of $N$ word tokens $(w_1, w_2,...,w_N)$. The commonly adopted way to encode the word tokens in hidden representations is to initially encode each word $w_t$ as a one-hot vector $[0,0,...,1,0,0]$ where the vector index that is $1$, corresponds to the word index $t$. Afterwards, each of the word vectors are embedded into $M$ dimensional vectors through a word embedding matrix $W_e^Tw_t$ \cite{nam2017dual, kiros2014unifying, wang2018learning, lee2018stacked, faghri2017vse++}. In this thesis, I argue that such an approach is undesired due to the lack of training data in the commonly used benchmark datasets. In order for the word embedding matrix to learn sufficiently good vector representations of words, it needs to be trained on much bigger datasets \cite{mikolov2013distributed, pennington2014glove, peters2018deep}. Moreover, I argue that intializing the word embedding matrix with a fixed pretrained word embeddings and then extracting a sentence meaning from this matrix\cite{klein2015associating} is undesired. This is the case because, in the problem of cross-modal retrieval, the text is a weak annotation of the image. In that sense, having fixed word embeddings limits the ability of the model to adjust the meaning of the word based on both the word context and matching image.
Therefore, instead of training the word embedding matrix from scratch, or initializing it with fixed word reprensetations I make use of a word embeddings from a language model, or also known as \textit{ELMo} \cite{peters2018deep}. Then, a higher level context depended representations of the words are learned with a \textit{GRU} recurrent neural network \cite{cho2014learning} on top of the \textit{ELMo}.\endgraf

The \textit{ELMo} word embeddings, are word embeddings obtained from the internal states of a bidirectonal langauge model. Given a sequence of $N$ word tokens $(w_1, w_2,...,w_N)$, a forward language model computes the probability of the whole sequence by computing the probability of a word token $w_t$, given its history $(w_1, w_2,...,w_{t-1})$:
$$p(w_1, w_2,...,w_N) = \prod_{t=1}^N p(w_t \vert w_1,w_2,...,w_{t-1})$$
On the other hand, a backward language model shares the same characteristics with a forward language model, except that it computes everything in reverse. In other words, it will compute the probability of the whole sequence by computing the probability of a word token $w_t$, given its future $(w_{t+1},w_{t+2},...,w_{N})$:
$$p(w_1, w_2,...,w_N) = \prod_{t=1}^N p(w_t \vert w_{t+1},w_{t+2},...,w_N)$$
\endgraf
The bidirectional language model used to compute \textit{ELMo}, is based on the recent state of the art language models \cite{melis2017state, jozefowicz2016exploring} where initially a context independent representation for each of the tokens is obtained using character convolutions. The obtained representation is then passed through \textit{L} layers of forward and backward \textit{LSTMs}\cite{hochreiter1997long}. In the model later used to compute \textit{ELMo}, during training, at time $t$, the \textit{LSTM} is presented with a context independent representation of the word token $w_t$. In order to convert the context independent representation of the word token $w_t$ into a context dependent one, the \textit{LSTM} goes through a multi step process that is managed by it's gates. The \textit{LSTM} has 3 of these gates. The first of the \textit{LSTM} gates decides how much of the past information is going to be forgotten:
$$f_t = \sigma(W_f \cdot [h_{t-1}, w_t] + b_f)$$
Because of the role that this gate plays, it is called the forget layer gate. The second step is to decide how much of the new information $w_t$ is going to be stored in the cell:
$$i_t = \sigma(W_i \cdot [h_{t-1}, w_t] + b_i)$$
$$\hat{C_t} = tanh(W_c \cdot [h_{t-1}, w_t] + b_c)$$
$$C_t = f_t * C_{t-1} + i_t * \hat{C_t}$$
Consequently, this gate is called the input layer gate because it updates memory of the cell based on the present information. Finally, the output layer gate is responsible for providing the context dependent representation for the word token $w_t$:
$$o_t = \sigma(W_o \cdot [h_{t-1}, w_t] + b_o)$$
$$h_t = o_t * tanh(C_t)$$
Here, 2 layers $(L = 2)$ of forward and backward \textit{LSTMs} are stacked on top of each other followed by a softmax that will return the probability for the next word in the sequence.\endgraf
In this thesis, in order to come up with \textit{ELMo} embeddings for each of the words tokens, the softmax layer is stripped off, and a weighted average of the internal states of the pretrained bidirectional language model\cite{peters2018deep} is computed. Assuming that we are given a sequence of $N$ tokens, then the \textit{ELMo} embedding of the sequence of tokens would be a tensor $E = (e_1^{elmo}, e_2^{elmo},...,e_N^{elmo})$ where each $e_t^{elmo}$ is a 1024 dimensional vector. However, the sequence representations $E$ still resides in the \textit{Bidirectional language model} latent space and the end goal is to have the text embedding projected in a joint latent space together with the image encoding. In order to do so, I add a multilayered bidirectional \textit{GRU}\cite{cho2014learning} on top. The purpose for having such a recurrent neural network on top of the \textit{ELMo} embeddings is besides projecting the sentence embedding in the image encoder space, to extract higher level context dependent representations for each of the words. In the \textit{image encoder} case for example, this was not needed and a simple fully connected layer is used to do the projection.\endgraf
The \textit{GRU} recurrent neural network, is a simplified version of the \textit{LSTM} \cite{hochreiter1997long} neural network, that fuses together the input and update layer gates into a single update gate. Consequently, given a sequence of past \textit{ELMo} word embeddings $E = (e_1^{elmo}, e_2^{elmo},...,e_{t-1}^{elmo})$, the hidden state of the forward $\overrightarrow{GRU}$ recurrent neural network, will be computed as result of the equations:


$$z_t = \sigma(W_z \cdot [h_{t-1}, e_t])$$
$$r_t = \sigma(W_r \cdot [h_{t-1}, e_t])$$
$$\hat{h_t} = tanh(W \cdot [r_t * h_{t-1}, e_t])$$
$$h_t = (1 - z_t) * h_{t-1} + z_t * \hat{h_t}$$

On the other hand, the backward $\overleftarrow{GRU}$ will obey to the same equations, with the only difference being that instead of considering the past hidden states, the hidden state $h_t$ will be conditioned on the future hidden states $\overleftarrow{H} = (h_N, h_{N-1},...,h_{t+1})$. Finally, in order to get the feature embedding for the whole text, I average together the forward and backward sequences of hidden states:

$$H = \frac{\overrightarrow{H} + \overleftarrow{H}}{2}$$


\begin{figure}
  \centering
  \includegraphics[width=45mm]{Images/sentence_encoder.png}
  \caption{The sentence encoder branch}
  \label{fig:sentence_encoder}
\end{figure}

On figure \ref{fig:sentence_encoder} the whole sentence encoder branch of the model is portrayed. The sequence of words $(w_1, w_2, w_3,...,w_n)$ are entering the \textit{Bidirectional language model} used to compute the \textit{ELMo} embeddings. The output $(E_1, E_2, E_3,...,E_n)$ are the \textit{ELMo} embeddings for each of the words in the sequence. After, each of \textit{ELMo} embeddings are passed through a bidirectional GRU recurrent neural network. Therefore, each of the final hidden states $(H_1, H_2, H_3,...,H_n)$ are the average of the forward and backward GRU cell where \textit{n} is the length of the sequence and the size of each $H_i$ correponds do the size of the joint space.

% ATTENTION %


\section{Attention block}
The attention mechanism, initially introduced by \cite{bahdanau2014neural}, is inspired by the human ability to pay attention to visual parts of an image or correlate words in a sequence\cite{rensink2000dynamic, corbetta2002control}. Firstly applied in the sequence-to-sequence translation with recurrent neural networks\cite{sutskever2014sequence} to overcome the encoder bottleneck, it was immediately extended to visual caption generation\cite{xu2015show} as well as document classification \cite{yang2016hierarchical}. A recently introduced variant of the attention mechanism is the self-attention \cite{cheng2016long} mechanism and it's extensions \cite{lin2017structured}. In this thesis, I am using a variant of \cite{lin2017structured, xu2015show} in order to attend multiple times at separate parts of both the image and the sentence. Moreover, I have empirically found that in case the weights of the attention block are reused across the modalities, the model achieves at least as good performance as the one that uses distinct attention weights for both of the modalities. Because such reusing of weights is firstly introduced by \cite{bromley1994signature} and their \textit{Siamese neural network}, I call this attention mechanism the \textit{Siamese multi-hop attention} or shortly \textit{SMHA}.\endgraf
\textit{SMHA} tries to leverage the fact that multiple entities are present within a sentence and an image, and for a model to be able to do the image-text and text-image retrieval successfully it has to attend multiple times on both modalities. By doing that, the outputs are attention weighed vectors where each vector encodes information about a distinct part of the image or the text. The \textit{SMHA} block is applied on top of both the image encoder and the text encoder and as such, it extracts these vectors for both modalities.\endgraf
Suppose that the output of one of the encoders consist of $N$ hidden representations $H = (H_1, H_2,H_3,...,H_N)$. Consequently, the dimensions of $H$ would be $(N, D)$ where $N$ are the number of hidden representations and $D$ is the dimensionality of the joint hidden space. However, in this case we are only ensured that \textit{D} is going to be the same for both of the outputs of the modality encoders. On the other hand, \textit{N} which is the number of tokens in the \textit{sentence encoder} case, and the spatial size in the \textit{image encoder} case, can vary from one modality to another. Therefore, in order to compute the similarity between the image embedding and the sentence embedding both of the modalities need to be encoded with a fixed size vector. An apparent method to overcome this issue is to perform a pooling on the hidden states from each of the modalities. However, when doing such pooling, all of the hidden states are going to contribute equally in the final embedding from each of the modalities which is undesirable. Therefore, the attention mechanism is a way to compute the final representation as a linear combination of the all hidden states. The atteniton mechanism takes the hidden states as inputs, and outputs a set of weights for each of the hidden states. The weights are computed using a one layer feed-forward neural network with a \textit{tanh} activation function. 

$$U = tanh(W_u H + b_u)$$
$$A_{t} = \frac{e^{U^T W_a}}{\sum_t e^{U^T W_a}}$$

In the equations above, $W_u$ is the weight matrix of the hidden layer and $W_a$ is the weight matrix of the output layer of the attention mechanism. The size of $W_u$ and $W_a$ are chosen as the values that yield the best score on the validation set. 

\begin{figure}
  \centering
  \includegraphics[width=45mm]{Images/attention_weights.png}
  \caption{Obtaining the attention weights}
  \label{fig:attention_weights}
\end{figure}

Figure \ref{fig:attention_weights} illustrates the process of coming up with the attention weights. Firstly the hidden representations $H$ are fed through a one layer multi-layered perceptron in order to extract a one level deeper hidden representation $U$. Then, this is followed by a softmax layer where the softmax operation is performed along the second dimension of the input.  The output of the softmax layer are the normalized attention weight for each of the attention hops.\endgraf
Once the attention weights are computed and normalized, the hidden states are weighted accordingly for each attention hop. 

$$O = A \cdot U$$

Figure \ref{fig:attention_attend} illustrates the weighting process. The matrix $O$ is the attended embedding matrix of the $H$ hidden representation, where each of the rows of the matrix are the outputs of a single attention hop.\endgraf

\begin{figure}
  \centering
  \includegraphics[width=45mm]{Images/attention_attend.png}
  \caption{Attending on the hidden states}
  \label{fig:attention_attend}
\end{figure}


In the case where the input to the attention block are the outputs from the image encoder, the attention weights are computed along the spatial dimension which is the $width \times height$ of the activation map of the last convolutional layer of the \textit{ResNet152}. On the other hand, if the input the to the attention block is the output from the sentence encoder, the attention weights are computed across the time dimension.\endgraf
Lastly, the the output of the attention block is normalized to lie on the unit hypersphere by computing:

$$O_{normalized} = O / \sqrt{max(\sum_i o_i^2,  \epsilon))}$$

Where $\epsilon$ is a small number to ensure that the element will be positive so that the square root can be defined in that domain.

\section{Matching loss}
In the supervised learning paradigm, namely classification, usually what we have is a fixed number of categories to which each of the instances of the training set belong. Namely, we are learning a mapping $X \rightarrow Y$ from a domain $X$ to a domain of target categories $Y$. Therefore, we can train a classifier that will minimize a loss function that indicates the mismatch between the target output $Y$ and the predicted output. However, there are cases where the number of categories is huge or in some cases even infinite. In those cases, we need to train a classifier that can recognize whether two instances belong to the same category or not.\endgraf
The triplet loss function, introduced by \cite{weinberger2009distance} is a way to train a classifier when the number of distinct categories can vary in the domain $[2, \infty]$. The goal of the triplet loss is to enforce one or more neural networks to encode two samples belonging to the same category close in the embedding space, and two sample belonging to a different category far apart in the embedding space. Therefore, in order to do so, the triplet loss employs three main components:
\begin{itemize}
    \item An anchor sample.
    \item A positive sample belonging to the same category as the anchor.
    \item A negative sample belonging to a category different than the anchor.
\end{itemize}

Based on the components, we can formalize the triplet loss as:

$$L = max(s(a,n) - s(a,p) + margin, 0.0)$$

Where the \textit{margin} is number that enforces the score between the anchor and the positive sample, to be higher than the score between the anchor and the negative sample by that number. The scoring function can be any function that returns the similarity between two vectors. In this case, $s(i,s)$ is a scoring function that computes the similarity between the image embedding $i$ and the sentence embedding $s$. The scoring function used in this thesis is the \textit{L2} normalized dot-product similarity which is the equivalent to the cosine similarity.

$$s(i,s) = \frac{i}{\vert \vert i \vert \vert^2} \cdot 
\frac{s}{\vert \vert s \vert \vert^2}$$

According to the loss function, we can define 3 types of triplets:
\begin{itemize}
    \item \textbf{Easy triplets}: Triplets where the loss is 0 because: $s(a,p) > s(a,n) + margin$
    \item \textbf{Semi-hard triplets}: Triplets where the loss is positive because of the influence of the margin: $s(a,n) < s(a,p) < s(a,n) + margin$
    \item \textbf{Hard triplets}: Triplets where the loss is positive even if we exclude the margin: $s(a,p) < s(a,n)$
\end{itemize}

Figure \ref{fig:hard_negatives} illustrates the three types of negative samples given an \textit{anchor} and a \textit{positive} sample. 

\begin{figure}
  \centering
  \includegraphics[width=60mm]{Images/hard_negatives.png}
  \caption{Easy, semi-hard and hard negatives}
  \label{fig:hard_negatives}
\end{figure}

\subsection{Offline vs online triplet mining}
The first challenge that occurs when using the \textit{triplet loss} coming up with the triplets. One way to mine the triplets would be to compute them at the start of a training epoch in an \textit{offline mining} way. However, the main problem with this approach is that it artificially increases the size of the training set with many redundant samples. Consequently, each of the triplets would have to be forward propagated through the neural network where many of the triplets would contain the same samples over and over again. In other words, if the batch size is $B$, the computational complexity would increase by a magnitude of $3B$. This results in an inneficient algorithm that is unnecessary slow. Moreover, with this approach, we have no control on the type if triplets that are going to be produced.\endgraf
On the other hand, the \textit{online mining} approach creates the triplets on the fly \cite{schroff2015facenet}. In the cross-modal retrieval case, a batch of pairs of matching images and sentences will propagated through the neural network in order to extract the embeddings for each image and sentence in the batch. Then, in each pair, the image or the sentence and interchangeably take the role of the anchor sample and the positive sample and the negative sample can be chosen within the batch. That would result in $2B$ instances propagated through the network, which eliminates the redundant computation present in the offline mining approach.

\subsection{Techniques for online mining the negative samples}
With the online mining method, we end up with $2B$ embeddings that correspond to a batch with $B$ pairs of an image and a sentence that match. On the other hand, the negatives are spread throughout the batch, and for each anchor image and a positive sentence, we have $B - 1$ negative sentences. Consequently, for each anchor sentence and a positive image, we have $B - 1$ negative sentences. The question that gets raised is what is an effective method pick the negatives. In that sense, there are two widely used strategies to select the negative samples, namely \textit{batch-all} and \textit{batch-hard} strategy.

\subsubsection{Batch-all mining of negatives samples}
In the \textit{batch-all} strategy for mining the negative samples the underling concept is that for each anchor image and a positive sentence, the loss will be summed over all negative sentences within the batch. Consequently, for each anchor sentence, and a positive image, the loss will be summed over all negative images within the batch. Based on the premise the loss that follows is:

$$L(i,t) = \sum_{\hat{t}}(s(i,\hat{t}) - s(i,t) + m) + \sum_{\hat{i}}(s(\hat{i},t) - s(i,t) + m)$$

The specific thing about the \textit{batch-all} loss is that the summation is done over both the semi-hard and the hard-negatives. Therefore, in \cite{faghri2017vse++} the triplet loss with \textit{batch-all} strategy is titled as \textit{sum of hinges loss}.  
\subsubsection{Batch-hard mining of negatives samples}
Inspired by \cite{hermans2017defense}, the authors in \cite{faghri2017vse++} make use of the \textbf{batch-hard} strategy to mine the negative samples. The motivation for doing so is that the success of the \textit{Recall@1} metric is dependent on the hardest negative within the batch.\endgraf
Therefore, based on \cite{faghri2017vse++}, the \textit{sum of hinges loss} can be redefined as:

$$L(i,t) = max_{\hat{t}}(s(i,\hat{t}) - s(i,t) + m)
+ max_{\hat{i}}(s(\hat{i},t) - s(i,t) + m)$$

To form the \textit{max of hinges loss} The main difference between the \textit{sum of hinges loss} and the \textit{max of hinges loss} is that with the \textit{max of hinges loss}, the hardest negative for each positive pair takes all the gradients, where in the \textit{sum of hinges loss} the gradients are computed over all the negatives with a batch \cite{faghri2017vse++}. Even though both the authors of \cite{faghri2017vse++} and \cite{lee2018stacked} report significant improvements by training only on the hardest negatives within a batch, I found that to be extremely unstable. Therefore, the experiments performed within this thesis are using the \textit{batch all strategy} to mine the negative samples.

\section{Attention weights diversification}
The obvious pitfall with using multiple hops of attention on the same sequence of vectors is the lack of diversity that can occur between the different hops. In other words, in the surface of the cost function shallow minimas may occur where all attention hops attend on the same part of the vector, and all but one attention hops become redundant. Therefore, as in \cite{lin2017structured} I include a penalization term that will enforce the attention weights to be diverse.\endgraf
Namely, what we want is each of attention hops to focus on encoding the least possible number of entities that are present within the image or mentioned within the sentence. Moreover, we want each of the attention hops to attend on a distinct part of the image or combination of words in the sentence. In other words, entity that no other attention hop has already attended to. Thus, the \textit{attention weight diversification} term that is added to the loss, and minimized together will the matching loss is:

$$L_{div} = \sum_i^M \sum_j^M \vert \vert A_iA_j^T - I \vert \vert^2$$

Where $M$ are the number of attention hops that are computed, $A_i$ and $A_j$ are two distinct attention hops and $I$ is the identity matrix.\endgraf
Because all of the attention weights for the different attention hops are normalized probabilities, they satisfy the constraint:

$$\sum_i^N a_i = 1$$

Where $N$ is the number of vectors that the model attends to. Consequently, the dot product between two distinct attention hops must satisfy the constraint:

$$0 =< a_i^T a_j =< 1$$

Where in the most extreme case, where both attention hops attend on the same thing the dot product is $1$. However, if we consider the reverse extreme case, where both attention weights attend on a unique entity the dot product is $0$, which is the desired outcome.\endgraf
Moreover, the elements on the main diagonal of $A_i A_j^T$ are the values when $a_i = a_j$. Therefore, they satisfy the above mentioned constraints as well. However, in this case, the desired outcome is that each of the attention hops to specialize in attending to the least possible number of entities. In other words, in case every single attention hop places its weight on exactly one entity the dot product will be $1$. As a result, I subtract the identity matrix from the equation which will enforce the specialization of attention hops. Finally, the Euclidean is computed for each of the attention hops. The $L_{div}$ is multiplied by an importance constant and minimized together with the matching triplet loss.

\section{Optimizer}
Once the total loss is computed the neural network needs to update its weights accordingly. There exist many different techniques to minimize the loss function where most of them are based on gradient descent\cite{ruder2016overview}. Gradient descent, in its purest form, computes the gradient of the cost function with respect to the neural network's parameters:

$$\theta = \theta - \alpha \cdot \nabla_{\theta} L(\theta)$$

Where $\theta$ are the network parameters, $\alpha$ is the learning rate and $L(\theta)$ is the loss function computed with respect to the model's parameters.\endgraf
The main issue with the default variant of gradient descent is that in order for the method to perform one update of the weights of the neural network, firstly the loss has to be computed for the entire training set at once. However, when dealing with large datasets that becomes unfeasible. Therefore, stochastic gradient descent \cite{robbins1951stochastic} is a simplified version of gradient descent, where the update formula becomes:

$$\theta = \theta - \alpha \cdot \nabla_{\theta} L(\theta; i^{z:z+n}; t^{z:z+n})$$

Based on the single weight update formula, stochastic gradient descent performs update on the weights after the loss is computed for a single batch. In the scope of this thesis, after the total loss is computed for a batch of images and sentences $(i^{z:z+n};t^{z:z+n})$, stochastic gradient descent is going to update the weights of the neural network. However, stochastic gradient descent encounters many challenges that need to be overcome in order for the neural network to converge in a sufficiently good minima. Some of these challenges are \cite{ruder2016overview}:

\begin{enumerate}
    \item The method is highly sensitive to the choice of the learning rate $\alpha$. Namely, a too high learning rate my result in never reaching convergence and a too low learning rate may take too much time to converge.
    \item The surface of the loss function is non-convex and has many saddle points where the gradient is $0$. As a result, gradient descent may get stuck in one of these points where no further update can be performed.
    \item Gradient descent uses the same learning rate to update all parameters in the model which is undesirable. Different parameters have different importance and updating them all the same is not a preferred property.
\end{enumerate}

The gradient descent optimizer with momentum \cite{qian1999momentum} is method that overcomes the \textit{2} challenge of stochastic gradient descent. The momentum optimizer update formula is:

$$v_t = \gamma v_{t-1} + \alpha \nabla_{\theta}L(\theta; i^{z:z+n}; t^{z:z+n})$$
$$\theta = \theta - v_t$$

Where a fraction $\gamma$ of the past update vector $v_{t-1}$ is added on the current update vector $v_t$ and $\gamma$ is a parameter that needs to be tuned on the validation set. However, even with the addition of the \textit{momentum term}, the challenges \textit{1} and \textit{3} are still present. In that sense, \textit{Adagrad's} \cite{duchi2011adaptive} update rule, at a time step $t$, for a single model parameter $\theta_i$ is:

$$\theta_{t+1,i} = \theta_{t,i} - \frac{\alpha}{\sqrt{G_{t, ii} + \epsilon}} \cdot \nabla_{\theta}L(\theta; i^{z:z+n}; t^{z:z+n})$$

Where $G_t$ is a diagonal matrix that stores the sum of squares of the gradients with respect to the parameter $\theta_i$ up to the time step $t$. The main advantage of \textit{Adagrad} is that the learning rate does not need to be tuned, and it is adapted for each parameter $\theta_i$ based on the past gradients for the parameter. However, \textit{Adagrad} comes together with the pitfall of rapidly decreasing learning rate. Namely, because the elements of $G$ are the sums of the squared gradients, as the training progresses, the learning rate becomes small to the point where the updates are insignificant. To overcome the vanishing learning rate, \textit{RMSprop} \cite{tieleman2012lecture} stores a running average of the magnitudes of past gradients in a fixed window $w$. Therefore, \textit{RMSprop} defines:

$$E[g^2]_t = \gamma E[g^2]_{t-1} + (1 - \gamma)g_t^2$$

where $g$ is the gradient of the loss function:

$$g_t = \nabla_{\theta}L(\theta; i^{z:z+n}; t^{z:z+n}$$

and $\gamma$ is defines the fraction of the past running average that will be considered. Therefore, the update formula for a single parameter $\theta_i$ becomes:

$$\theta_{t+1,i} = \theta_{t,i} - \frac{\alpha}{\sqrt{E[g^2]_t + \epsilon}} \cdot g_t$$

A final enhancement of gradient descent is \textit{Adam} \cite{kingma2014adam}. \textit{Adam} is a method that is built on top of \cite{tieleman2012lecture, duchi2011adaptive} that also keeps a exponentially decaying average of past gradients which is similar to the momentum term in \cite{qian1999momentum}. Therefore, the exponentially decaying average of past gradients like the gradient descent optimizer with \textit{momentum} is:

$$m_t = \beta_1 m_{t-1} + (1 - \beta_1)g_t$$

While the exponentially decaying average of past squared gradients like \textit{RMSprop} is:

$$v_t = \beta_2 v_{t-1} + (1 - \beta_2)g_t^2$$

Based on the equation defined above, the update rule for a single parameter $\theta_i$ of the \textit{Adam} optimizer becomes:

$$\theta_{t+1,i} = \theta_{t,i} - \frac{\alpha}{\sqrt{v_t + \epsilon}} \cdot m_t$$

In \cite{kingma2014adam} it is empirically proven that \textit{Adam} is guaranteed to perform as least as good as the other simpler variants so it is the go to optimizer in the scope of this thesis.

\section{Weights intialization}
NA KRAJ MOZES I NESTO DA NAPISES ZA OVA
%%% Local Variables: 
%%% mode: latex
%%% TeX-master: "thesis"
%%% End: 

\chapter{Experiments}
\label{experiments}

\section{Datasets}
Throughout this thesis, I perform experiments on the \textit{Pascal1k} \cite{rashtchian2010collecting}, \textit{Flickr8k} \cite{hodosh2013framing}, and \textit{Flickr30k} \cite{young2014image} datasets. All of them contain images collected from \textit{Flickr}, where 5 annotating sentences accompany each image. The \textit{Pascal1k} dataset contains 1000 images where no particular splits are provided. Because the 1000 images are spread throughout 20 different categories, I extract 40 images together with their sentences from each category for training. The remaining 10 images per category together with the corresponding sentences are split as 5 images per category for validation and 5 images per category for testing as done by \citet{klein2015associating, socher2014grounded, frome2013devise, karpathy2014deep}. In total, 800 images are used for training, 100 for validation, and 100 as a test set. The \textit{Flickr8k} dataset provides concrete splits for training, validation, and testing. In total, there are 6091 images for training, 1000 images for validation, and 1000 images for testing. Lastly, the \textit{Flickr30k} dataset also provides training, validation, and testing splits \cite{young2014image} where 29783, 1000, and 1000 are used for training, validation, and testing accordingly.

\section{Experimental setup}
To evaluate the \textit{Siamese multi-hop attention} model, I conduct extensive experiments and report the \textit{Recall@K} metric, where \textit{K} is taken to be \textit{1, 5}, or \textit{10}. That is, given a dataset of images and their corresponding sentences,  the purpose is to retrieve a ranked list of sentences for each image (image-text retrieval) or to retrieve a ranked list of images for each sentence (text-image retrieval). Then, the sentence retrieval \textit{Recall@K} would be the number of times the correct sentence is within the top \textit{K} entries of the ranked list when the ranked list is computed for each image. On the other hand, the image retrieval \textit{Recall@K} would be the number of times the correct image is within the top \textit{K} entries of the ranked list when the ranked list is computed for each sentence. All hyperparameters of \textit{SMHA} are selected as the values that yield the best \textit{Recall@1} on the validation set, and the trained model is used for inference on the test set. A detailed overview of the values chosen for the final models is presented in the appendix section \ref{app: training}. Finally, in order to overcome the overfitting pitfall when training deep neural networks, during training, the model weights are saved only when \textit{Recall@1} on the validation set improved compared to the previous epochs. 

\section{Quantitative evaluation}
\subsection{Pascal1k}
\begin{table}[H]
  \centering
  \begin{tabular}{@{}lccccccc@{}} \toprule
    \multicolumn{1}{c}{} &
    \multicolumn{3}{c}{Sentence retrieval} &
    \multicolumn{3}{c}{Image retrieval} \\
    \cmidrule(r){2-7}
    Method  & R@1 &  R@5 & R@10 & R@1 & R@5 & R@10 \\ \midrule
    SDT-RNN \cite{socher2014grounded}   &    25.0     &      56.0    &      70.0    &   25.4      &    65.2     &     84.4     \\
    DFE \cite{karpathy2014deep}  &    39.0     &      68.0    &      79.0    &   23.6      &    65.2     &     79.8     \\
    FV \cite{klein2015associating}   &    \textbf{55.9}     &      \textbf{86.2}    &      93.3    &   \textbf{44.0}      &    \textbf{85.6}     &     94.6     \\
    Devise \cite{frome2013devise}  &    17.0     &      57.0    &      68.0    &   21.6      &    54.6     &     72.4     \\ \midrule
    SMHA  &    44.0     &      80.0    &      \textbf{95.0}    &   40.0      &    80.0     &     \textbf{96.0}     \\ \bottomrule
  \end{tabular}
  \caption{Results on the Pascal sentences dataset.}
  \label{pascal1kresults}
\end{table}
On the Pascal1k dataset, as presented on table \ref{pascal1kresults}, we can see that \citet{klein2015associating} outperform the other methods on the \textit{Recall@1} and \textit{Recall@5} metrics. However, on the \textit{Recall@10} metric, the \textit{SMHA} reports better results than \textit{FV}.\endgraf
The fact that the \textit{FV} method outperforms all the rest is related to \textit{Occam's razor} principle in machine learning. The \textit{Pascal1k} dataset is rather small, and methods that rely on deep learning can easily overfit the training data. On the other hand, \textit{DFE} \cite{karpathy2014deep},  \textit{SDT-RNN} \cite{socher2014grounded}, and \textit{Devise} \cite{frome2013devise} due to their simple structure, can not properly fit the data and report results much lower than \textit{SMHA} on all metrics.
\subsection{Flickr8k}
\begin{table}[H]
  \centering
  \begin{tabular}{@{}lccccccc@{}} \toprule
    \multicolumn{1}{c}{} &
    \multicolumn{3}{c}{Sentence retrieval} &
    \multicolumn{3}{c}{Image retrieval} \\
    \cmidrule(r){2-7}
    Method  & R@1 &  R@5 & R@10 & R@1 & R@5 & R@10 \\ \midrule
    SDT-RNN \cite{socher2014grounded}   &    6.0     &      22.7    &      34.0    &   6.6      &    21.6     &     31.7     \\
    Devise \cite{frome2013devise}  &    4.8     &      16.5    &      27.3    &   5.9      &    20.1     &     29.6     \\
    FV \cite{klein2015associating}   &    31.0     &      59.3   &      73.6    &   21.2      &    \textbf{50.0}     &     64.8     \\
    DFE \cite{karpathy2014deep}  &    12.6     &      32.9   &      44.0    &   9.7      &    29.6     &     42.5     \\ 
    m-RNN \cite{mao2014explain}  &    14.5     &      37.2    &      48.5    &   11.5      &    31.0     &     42.4     \\ 
    DVSA \cite{karpathy2015deep}  &    16.5     &      40.6    &      54.2    &   11.8      &    31.8     &     44.7     \\ 
    MNLM \cite{kiros2014unifying}  &    18.0     &      40.9    &      55.0    &   12.5      &    37.0     &     51.5     \\ 
    m-CNN \cite{ma2015multimodal}  &    15.6     &      40.0    &      55.7    &   14.5      &    38.2     &     52.6     \\ \midrule
    SMHA  &    \textbf{33.0}     &      \textbf{61.0}    &      \textbf{75.0}    &   \textbf{22.0}      &    \textbf{50.0}     &     \textbf{66.0}     \\ \bottomrule
  \end{tabular}
  \caption{Results on the Flickr8k dataset.}
  \label{flickr8kkresults}
\end{table}

Because of the increased size of the \textit{Flickr8k} dataset compared to the \textit{Pascal1k} dataset, methods that rely on deep learning \cite{ma2015multimodal, kiros2014unifying, karpathy2015deep, mao2014explain, klein2015associating}, are starting to significantly outperform simple methods \cite{socher2014grounded, frome2013devise, karpathy2014deep}. This can be seen on table \ref{flickr8kkresults}. Also, due to the increased size of the dataset, \textit{SMHA} is taking over the first place from \textit{FV}. However, this dataset can still be considered as small compared to the amount of data typically required for deep learning models to perform well and yet again, a simple method may yield competitive results.

\subsection{Flickr30k}
\begin{table}[H]
  \centering
  \begin{tabular}{@{}lccccccc@{}} \toprule
    \multicolumn{1}{c}{} &
    \multicolumn{3}{c}{Sentence retrieval} &
    \multicolumn{3}{c}{Image retrieval} \\
    \cmidrule(r){2-7}
    Method  & R@1 &  R@5 & R@10 & R@1 & R@5 & R@10 \\ \midrule
    MNLM \cite{kiros2014unifying}   &    23.0     &      50.7    &      62.9    &   16.8      &    42.0     &     56.5     \\
    DAN \cite{nam2017dual}  &    55.0     &      81.8    &      89.0    &   39.4      &    69.2     &     79.1     \\
    VSE++ \cite{faghri2017vse++}   &    43.7     &       71.9     &      82.1    &   32.3        &    60.9     &     72.1     \\
    VSE++ (Fine-tuned) \cite{faghri2017vse++}   &    52.9     &      80.5    &      87.2    &   39.6      &    70.1     &     79.5     \\
    sm-LSTM \cite{huang2017instance}  &    42.5     &      71.9    &      81.5    &   30.2      &    60.4     &     72.3     \\ 
    m-CNN \cite{ma2015multimodal} &    33.6     &      64.1    &      74.9    &   26.2      &    56.3     &     69.6  \\    
    FV \cite{klein2015associating} &    35.0     &      62.0    &      73.8    &   25.0      &    52.7     &     66.0     \\ 
    SCAN (Fine-tuned) \cite{lee2018stacked} &  \textbf{67.4} & \textbf{90.3} & \textbf{95.8} & \textbf{48.6} & \textbf{77.7} & \textbf{85.2}    \\ 
    EmbeddingNet (Fine-tuned) \cite{wang2018learning} &    40.7     &      69.7    &      79.2    &   29.2      &    59.6     &     79.7     \\ \midrule
    SMHA &    45.5     &      75.2    &      83.2    &   32.1      &    63.5     &     74.3     \\ \bottomrule
  \end{tabular}
  \caption{Results on the Flickr30k dataset}
  \label{flickr30kkresults}
\end{table}
As we can see in table \ref{flickr30kkresults}, \textit{SMHA} reports competitive results compared to the other methods. Moreover, \textit{SMHA} is compared to methods where the image encoder is fine-tuned as well as methods where it is kept fixed. From the table, it is evident that fine-tuning the image encoder is crucial for doing successful image-text matching on the \textit{Flickr30k} dataset. An example is the relative improvement of 10\% on the \textit{Recall@1} and \textit{Recall@5} metrics that the \textit{VSE++} \cite{faghri2017vse++} achieves by fine-tuning the image encoder.\endgraf
Due to of the increased size of the \textit{Flickr30k} dataset, the more sophisticated many-to-many matching methods \cite{nam2017dual, lee2018stacked, huang2017instance} start to significantly outperform the one-to-one matching methods \cite{ma2015multimodal, klein2015associating, kiros2014unifying, wang2018learning}. \textit{SCAN} \cite{lee2018stacked}, in particular, due to its fine-grained methodology, achieves \textit{SOTA} results on all metrics and outperforms other methods by 10\% on both \textit{Recall@1} and \textit{Recall@5}. As mentioned above, considering the relative improvement of 10\% that \textit{VSE++} achieves by fine-tuning the image encoder, it can be derived that by fine-tuning the image encoder of \textit{SMHA}, the results can be improved significantly.

\section{Qualitative evaluation}

By visualizing the attention weights from each attention hop, we can see the image and sentence elements that played a significant role when the modalities are embedded in the latent space. Furthermore, the ability to visualize the attention process adds an interpretability feature to the model which many of the other non-attention based approaches lack. Besides, this demonstrates that the multi-hop attention module learns to attend on modality segments which are naturally aligned.\endgraf

Figure \ref{fig:image-text-1} depicts the image-text retrieval alongside a visualization of the attention weights from two randomly picked attention hops. As we can see on the first visual attention hop, the main focus is the woman who is in the foreground of the image. The semantic attention behaves similarly. The highest-ranked retrieved sentence "an asian boy and an asian girl are smiling in a crowd of people" mostly highlights the phrases "asian boy" and "asian girl". The other words that have an increased weight are "smiling", "crowd", and "people". All of these words have a less significant role in the sentence meaning, and thus, their weight compared to the two main word entities is smaller. On the other hand, the second attention hop portrays a more general overview of the image and the sentence. The visual attention is spread throughout most of the image and it is dominant in the background. The semantic attention behaves accordingly. The words that have the biggest weight are "people" and "crowd", which is in accordance with the visual attention.\endgraf

\begin{figure}
  \centering
  \includegraphics[width=130mm]{Images/image-text.pdf}
  \caption[Image-text retrieval]{This figure visualizes the image-text retrieval. The image on the left is the query image while the images on the right are the attention weights from two distinct visual attention hops. The 5 sentences below the images are the retrieved sentences. The red-colored words indicate increased attention to those words.}
  \label{fig:image-text-1}
\end{figure}


Figure \ref{fig:text-image}, exhibits the text-image retrieval on a randomly chosen query sentence from the \textit{Flickr8k} dataset, together with two arbitrary attention hops. Due to the rather short sentence, both attention hops put identical weight intensity on the same words. Specifically, the increased attention weights are on the most meaningful words such as "hiker", "snowy", and "hill". Additionally, all of the retrieved images contain snowy hills and a person hiking on the hills, which is consistent with the query sentence. The visual attention, unlike the semantic attention, maintains a slight diversity across the two attention hops. The attention hops on some of the retrieved images play a different role where one attention hop attends more on the hiker and less on the background, whereas the other is doing the exact opposite. However, in most of the images, both of the attention hops attend on the same area of the image. This indicates that the real benefits of multi-hop attention over single-hop attention, are becoming more apparent when dealing with fine-grained use cases with multi-entity contents.

\begin{figure}
  \centering
  \includegraphics[width=110mm]{Images/text-image.pdf}
  \caption[Text-image retrieval]{This figure visualizes the text-image retrieval. The sentence represents the query where the words with a red background indicate increased attention on those words. The 5 images in the first column are the top 5 retrieved images. The 5 images in the middle column and the 5 images in the column on the right represent the attention weights on the retrieved images from two distinct attention hops.}
  \label{fig:text-image}
\end{figure}

Figure \ref{fig:image_attn_1} depicts the visual attention across 11 distinct attention hops. As we can see, the raw image consists of two different coloured dogs that are playing with a frisbee. Because of the fine-grained nature of the image, we can observe that the first three visual attention hops (the second, third and fourth image), attend on the general outline of the image, or in other words, the attention weights are spread on all 3 entities. The fourth, fifth, sixth and seventh attention hop (the 4 images in the second row) attend mainly on the black dog and the frisbee. Lastly, the final four attention hops (the 4 images in the third row) attend on the brown dog.

\begin{figure}
  \centering
  \includegraphics[width=110mm]{Images/image_attn_80_index.png}
  \caption[Visual multi-hop attention on a fine-grained image]{This figure illustrates the visual multi-hop attention across 11 distinct attention hops. The original image is the one on the top-left while the 11 attention hops are the images with the heat-map overlay. Since the image is fine-grained, separate attention hops attend on different, yet correlated segments of the image.}
  \label{fig:image_attn_1}
\end{figure}

Figure \ref{fig:image_attn_2} showcases a scenario where the primary entity which is the basketball player dominates the image. Consequently, we can observe that the visual attention is predominant around that entity with little to no diversity between different attention hops. Namely, as stated above, the real benefit of the multi-hop attention remains non-utilized on image-text samples with single-entity structure.\endgraf

\begin{figure}
  \centering
  \includegraphics[width=110mm]{Images/image_attn_190_index.png}
  \caption[Visual multi-hop attention on a non-complex image]{This figure illustrates the visual multi-hop attention across 11 distinct attention hops. The original image is the one on the top-left while the 11 attention hops are the images with the heat-map overlay. Because of the modest image contents, there is a lack of diversity between the attention hops.}
  \label{fig:image_attn_2}
\end{figure}

\section{Ablation studies}
\subsection{Effect of using multiple attention hops}
Compared to the additive attention \cite{bahdanau2014neural} and its later extension the visual attention \cite{xu2015show}, the attention mechanism used in this thesis extends the single-hop attention to a multi-hop one. The motivation for going from a single hop of attention to multiple hops of attention is related to the fine-grained structure of images and annotating sentences. Therefore, in this section ablation studies are performed to demonstrate the necessity of having multiple attention hops when doing image-text matching.\endgraf
Figure \ref{fig:attn_hops} showcases two models that are trained for 10 epochs on the \textit{Flickr8k} dataset and \textit{Recall@1}, \textit{Recall@5}, and \textit{Recall@10} is reported for each epoch on the test set. As we can see, regardless of the metric, the model that leverages multiple hops of attention outperforms the model that uses a single hop. This suggests superiority of multi-hop attention \cite{lin2017structured} compared to the single-hop one \cite{bahdanau2014neural, xu2015show} in the image-text matching problem.

\begin{figure}
  \centering
  \includegraphics[width=120mm]{Images/single_hop_vs_multi_hop.png}
  \caption[Comparison between single-hop attention and multi-hop attention]{Comparison of two models with random configuration, where one of the models uses single-hop attention and the other uses multi-hop attention. Both models are trained for 10 epochs on the \textit{Flickr8k} dataset while their \textit{Recall@1}, \textit{Recall@5} and \textit{Recall@10} is reported on the test set. The model that leverages multi-hop attention outperforms the model that uses single-hop attention on all metrics.}
  \label{fig:attn_hops}
\end{figure}

\subsection{Effect of the attention weights diversification term}
The problem that arises when using multi-hop attention is redundancy between attention hops. The purpose of having multiple attention hops can be considered useless in case all attention hops provide highly correlated attention weights. Another explanation of the redundancy can be that shallow saddle points appear on the surface of the loss function, where the model is not fully utilizing its potential. To overcome this issue, a term to maintain diversity between the attention hops is implemented.\endgraf

On figure \ref{fig:attn_diverse}, two models that employ multi-hop attention are trained on the \textit{Flickr8k} dataset for 10 epochs and their \textit{Recall@1}, \textit{Recall@5} and \textit{Recall@10} is reported on the test set. The difference between the models is that one of the models enforces the attention weights to be diverse and the other does not. As we can see in the figure, on all 3 metrics, the model that minimizes the diversification term together with the matching loss slightly outperforms the model that does not.

\begin{figure}
  \centering
  \includegraphics[width=120mm]{Images/attn_diverse_vs_non_diverse.png}
  \caption[Comparison between a model trained with and without attention weights diversification]{Comparison of two models sharing the same configuration where one of the models is trained to minimize the matching loss only while the other minimizes the attention weights diversification term together with the matching loss. Both models are trained for 10 epochs on the \textit{Flickr8k} dataset while their \textit{Recall@1}, \textit{Recall@5} and \textit{Recall@10} is reported on the test set. The plot of the metrics supports the premise that the attention weights diversification terms help the model to fully utilize the learning capacity.}
  \label{fig:attn_diverse}
\end{figure}

Moreover, figure \ref{fig:2_hop_frob} illustrates the outputs of two visual attention hops when the model is trained to enforce diverse attention weights. As we can see, each of the two distinct visual attention hops either specializes on a certain feature of the image or extends the attended part of the previous attention hop. In this particular case, the first attention hop (the second image from the left), attends mainly on the red ribbon and on the man on the left. The second attention hop (the third image from the left), extends the first attention hop and attends on the background. 

\begin{figure}
  \centering
  \includegraphics[width=120mm]{Images/2_hop_frob.png}
  \caption[Visual attention of a model trained with attention weights diversification]{Visual attention when the model is trained together \textbf{with} minimizing a term to ensure diversity between the attention hops.}
  \label{fig:2_hop_frob}
\end{figure}

On the contrary, figure \ref{fig:2_hop_no_frob} depicts a model that is trained without enforcing diversity between the attention weights. Here, the attention weights are correlated, which is the opposite of figure \ref{fig:2_hop_frob}. Both the first and the second attention hops have put equivalent attention on the ribbon and the background. Similarly, the attention put on the two people is almost indistinguishable between the two attention hops.

\begin{figure}
  \centering
  \includegraphics[width=120mm]{Images/2_hop_no_frob.png}
  \caption[Visual attention of a model trained without attention weights diversification]{Visual attention when the model is trained \textbf{without} minimizing a term to ensure diversity between the attention hops.}
  \label{fig:2_hop_no_frob}
\end{figure}

\subsection{Effect of using siamese attention}

The \textit{Siamese neural network} \cite{bromley1994signature} consists of two or more copies of a single neural network that have tied weights. During the ablation studies, I have empirically found that tying the weights of the attention module, even though the inputs come from a completely different source of information does not hurt the performance of the model. On figure \ref{fig:siamese_vs_non_siamese} two models, one using \textit{siamese attention} and one using separate weights for the attention modules are trained for 10 epochs on the \textit{Flickr8k} dataset and their \textit{Recall@1}, \textit{Recall@5} and \textit{Recall@10} is reported on the test set. From the figure, we can see that the model with tied attention weights performs at least as good as the one that uses separate weights. This suggests that the attention module "learns to pay attention" and can be easily reused across modalities.

\begin{figure}
  \centering
  \includegraphics[width=120mm]{Images/siamese_vs_non_siamese.png}
  \caption[Comparison between tied and untied visual-semantic multi-hop attention]{This figure represents a comparison between a model that has tied attention weights for both modalities and a model with distinct attention weights for each of the modalities. Both models are trained for 10 epochs on the \textit{Flickr8k} dataset, and \textit{Recall@1}, \textit{Recall@5}, and \textit{Recall@10} is reported on the test set for each epoch. The less memory demanding model with tied attention weights for both the visual and textual attention module achieves performance at least as good as the one with untied attention weights on all metrics.}
  \label{fig:siamese_vs_non_siamese}
\end{figure}

\subsection{Effects of using \textit{ELMo}}\label{sec: elmo_results}

As mentioned in section \ref{sec: sentence_encoder}, in this thesis transfer learning is applied on both the image encoder branch, as a pre-trained \textit{ResNet152} \cite{he2016deep} and on the sentence encoder branch, as a pre-trained bidirectional language model to obtain \textit{ELMo} \cite{peters2018deep}. To the best of my knowledge, compared to other methods for image-text matching where transfer learning is applied on the image encoder branch, this work is the first one to use \textit{ELMo} \cite{peters2018deep} on the sentence encoder branch. The motivation for transferring knowledge to the sentence encoder is due to the extreme data scarcity of the image-text matching datasets. Namely, in order for a deep learning model to learn sufficiently good word embeddings, it has to be trained on much bigger datasets \cite{mikolov2013distributed, peters2018deep}. Therefore, figure \ref{fig:elmo_vs_no_elmo} illustrates two models trained on the \textit{Flickr8k} dataset for 10 epochs while \textit{Recall@1}, \textit{Recall@5} and \textit{Recall@10} is reported on the test set.

\begin{figure}
  \centering
  \includegraphics[width=120mm]{Images/elmo_vs_non_elmo.png}
  \caption[Comparison between a model that leverages ELMo and a model that learns word embeddings from scratch]{This figure compares two models, one that makes use of pre-trained \textit{ELMo} \cite{peters2018deep} and another model that trains a word embedding matrix from scratch. Both models are trained for 10 epochs on the \textit{Flickr8k} dataset and their \textit{Recall@1}, \textit{Recall@5} and \textit{Recall@10} is reported on the test set. The model that leverages pre-trained \textit{ELMo} significantly outperforms the mode that learns the word embeddings from scratch.}
  \label{fig:elmo_vs_no_elmo}
\end{figure}

As we can see in figure \ref{fig:elmo_vs_no_elmo}, the model that uses \textit{ELMo} drastically outperforms the one that learns the word embeddings from scratch. As mentioned above, this is due to insufficient amount of training data compared to how much data is needed to learn sufficiently good word embeddings. On the other hand, \textit{ELMo} are obtained from a bidirectional language model trained on the 1 Billion words dataset \cite{chelba2013one}, which is a dataset several orders of magnitude more prominent than the ones used to train the \textit{Siamese multi-hop attention} model.

%%% Local Variables: 
%%% mode: latex
%%% TeX-master: "thesis"
%%% End: 

\chapter{Conclusion}
\label{cha:conclusion}
In this thesis, I propose \textit{Siamese multi-hop attention}, a deep learning architecture for image-text matching. My main contribution is extending the work of \citet{lin2017structured} for the image-text matching problem. I carry out a comprehensive empirical analysis to emphasize that using multiple hops of attention is essential for image-text matching. By visualizing the image-text matching, we can conclude that the model learns to attend on the essential parts of the image and the sentence while neglecting the rest. Moreover, I propose using a variant of the multi-hop attention called \textit{siamese multi-hop attention}, where the visual and textual attention have tied weights. In addition to that, I conduct experiments to prove that the less memory demanding siamese variant is guaranteed to perform at least as good as the one that uses separate weights for the visual and textual attention. Consequently, all experiments conducted use the \textit{siamese multi-hop attention}. This raises an opportunity to extend this work in a direction where the knowledge from pre-trained attention weights is transferred across modalities to achieve multimodal transfer learning.\endgraf
The industrial application of such deep learning architecture is incredibly broad. So far, the industry is flooded with algorithms that can perform information retrieval in the text-text or image-image cases. However, to the best of my knowledge, a cross-modal retrieval algorithm is yet to be applied in a commercial application. Therefore, the \textit{Siamese multi-hop attention} deep learning architecture can operate within any information retrieval system where the queries submitted by the users are textual, and the database entries are images or vice versa.
%%% Local Variables: 
%%% mode: latex
%%% TeX-master: "thesis"
%%% End: 


% If you have appendices:
\appendixpage*          % if wanted
\appendix
\chapter{Details of training}
\label{app: training}
The whole end-to-end deep learning architecture is implemented using the library Tensorflow \cite{abadi2016tensorflow}. Moreover, to come up with the best hyperparameter combination for each of the datasets, hyperparameter tuning is performed with the library Hyperopt \cite{bergstra2015hyperopt}.\endgraf
To train the models in this thesis, I use the Adam \cite{kingma2014adam} optimizer with a starting learning rate of 0.0006 for training on the \textit{Pascal1k} and \textit{Flickr8k} datasets, and a starting learning rate of 0.0004 for training on the \textit{Flickr30k} dataset. During training, the learning rate is halved down every 5 epochs for the smaller \textit{Pascal1k} and \textit{Flickr8k} datasets and every 10 epochs for the larger \textit{Flickr30k} dataset. A margin of 0.2 is used for training a \textit{Siamese multi-hop attention} model on all three datasets. The number of neurons of the hidden layer for the multi-hop attention is set to 256 for the \textit{Pascal1k} and \textit{Flickr8k} dataset, and 64 neurons are used for the \textit{Flickr30k} dataset. In addition, 30 attention hops are used for training a model on the Pascal1k and \textit{Flicrk8k} datasets, and 10 attention hops for the \textit{Flickr30k} dataset. The size of the joint space where both modalities are projected is set to 1024 and the gradients are clipped when the global norm exceeds 2.0 for all three datasets. The penalization term is used as it is, meaning that is not scaled by a constant. Moreover, an L2 weight decay of 0.0001 is used in addition to the matching loss and the penalization term to contribute to the total model loss.
%%% Local Variables: 
%%% mode: latex
%%% TeX-master: "thesis"
%%% End: 

\chapter{Additional examples}
\label{additional_examples}
In this section I present additional examples of cross-modal retrieval on the image-text and text-image cases. In particular, I showcase the query modality as well as the top 5 retrieved samples from the opposing modality. Moreover, for each of the modalities the attention weights are visualized.

\begin{figure}
  \centering
  \includegraphics[width=130mm]{Images/image-text-pascal.pdf}
  \caption[Image-text retrieval \textit{Pascal1k}]{Visualizing the image-text retrieval on the \textit{Pascal1k} dataset. The image on the left is the query image while the images on the right are the attention weights from two distinct attention hops. The 5 sentences below the images are the retrieved sentences. The red colored words indicates increased attention weights on those words.}
  \label{fig:image-text-pascal}
\end{figure}

\begin{figure}
  \centering
  \includegraphics[width=120mm]{Images/text-image-pascal.pdf}
  \caption[Text-image retrieval \textit{Pascal1k}]{Visualizing the text-image retrieval on the \textit{Pascal1k} dataset. The image on the left is the query image while the images on the right are the attention weights from two distinct attention hops. The 5 sentences below the images are the retrieved sentences. The red colored words indicates increased attention weights on those words.}
  \label{fig:text-image-pascal}
\end{figure}

\begin{figure}
  \centering
  \includegraphics[width=130mm]{Images/image-text-flicrk8k.pdf}
  \caption[Image-text retrieval \textit{Flickr8k}]{Visualizing the image-text retrieval case on the \textit{Flickr8k} dataset. The image on the left is the query image while the images on the right are the attention weights from two distinct attention hops. The 5 sentences below the images are the retrieved sentences. The red colored words indicates increased attention weights on those words.}
  \label{fig:image-text-flickr8k}
\end{figure}

\begin{figure}
  \centering
  \includegraphics[width=120mm]{Images/text-image-flickr8k.pdf}
  \caption[Text-image retrieval \textit{Flickr8k}]{Visualizing the text-image retrieval on the \textit{Flickr8k} dataset. The image on the left is the query image while the images on the right are the attention weights from two distinct attention hops. The 5 sentences below the images are the retrieved sentences. The red colored words indicates increased attention weights on those words.}
  \label{fig:text-image-flickr8k}
\end{figure}

\begin{figure}
  \centering
  \includegraphics[width=130mm]{Images/image-text-flickr30k.pdf}
  \caption[Image-text retrieval \textit{Flickr30k}]{Visualizing the image-text retrieval case on the \textit{Flickr30k} dataset. The image on the left is the query image while the images on the right are the attention weights from two distinct attention hops. The 5 sentences below the images are the retrieved sentences. The red colored words indicates increased attention weights on those words.}
  \label{fig:image-text-flickr30k}
\end{figure}


\begin{figure}
  \centering
  \includegraphics[width=120mm]{Images/text-image-flickr30k.pdf}
  \caption[Text-image retrieval \textit{Flickr30k}]{Visualizing the text-image retrieval on the \textit{Flickr30k} dataset. The image on the left is the query image while the images on the right are the attention weights from two distinct attention hops. The 5 sentences below the images are the retrieved sentences. The red colored words indicates increased attention weights on those words.}
  \label{fig:text-image-flickr30k}
\end{figure}

%%% Local Variables: 
%%% mode: latex
%%% TeX-master: "thesis"
%%% End: 


\backmatter
% The bibliography comes after the appendices.
% You can replace the standard "abbrv" bibliography style by another one.
\bibliographystyle{abbrv}
\bibliography{references}

\end{document}

%%% Local Variables: 
%%% mode: latex
%%% TeX-master: t
%%% End: 
